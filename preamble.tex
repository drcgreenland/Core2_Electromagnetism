\usepackage{booktabs}
%\documentclass[11pt]{book}[pdfborder={0 0 0}]
\usepackage[a4paper,margin=3cm]{geometry}
\usepackage[colorlinks]{hyperref}
\usepackage[svgnames]{xcolor}
\usepackage{bm,graphicx,enumitem,mathtools}
%\usepackage{quantum}
\usepackage{empheq}
\usepackage{framed,float,pdfcomment,tcolorbox}

%\usepackage{ms}

%\usepackage[utf8]{inputenc}

\usepackage[margin=20pt,font=footnotesize,labelfont=bf,labelsep=endash]{caption}

\usepackage[T1]{fontenc}

%\usepackage{newtxtext,newtxmath}

%\usepackage[cal=boondoxo]{mathalfa}

\usepackage[defaultsans]{lato}
\usepackage{esint}
\usepackage{booktabs} 

\usepackage[titles]{tocloft}
%\renewcommand*{\cftsecdotsep}{4.5}  % use dots in the section entries
\renewcommand*{\cftsecnumwidth}{2em} % increase space for Roman numerals
\renewcommand*{\cftsubsubsecindent}{2em} % reduce indent of subsubsection titles 
\setlength\cftparskip{4pt}

\renewcommand\thesection{\arabic{section}}

\renewcommand\contentsname{\sffamily\Large Table of Contents\medskip\hrule\medskip}

\usepackage[nodayofweek]{datetime}

\makeatletter
\renewcommand\section{%
\@startsection{section}{1}{\z@}%
              {-2ex \@plus -1ex \@minus -.2ex}%
              {1ex \@plus .2ex}%
              {\sffamily\bfseries\large\noindent Section~}}
\renewcommand\subsection{%
\@startsection{subsection}{2}{\z@}%
              {-3.25ex\@plus -1ex \@minus -.2ex}%
              {1ex \@plus .2ex}%
              {\sffamily\bfseries}}
\renewcommand\subsubsection{%
\@startsection{subsubsection}{2}{\z@}%
              {-3.25ex\@plus -1ex \@minus -.2ex}%
              {1ex \@plus .2ex}%
              {\sffamily\bfseries}}
\renewcommand\paragraph{%
\@startsection{paragraph}{4}{\z@}%
              {-1.5ex\@plus -1ex \@minus -.2ex}%
              {0.3ex \@plus .2ex}%
              {\sffamily\bfseries}}

\renewcommand\tableofcontents{%
    \subsection*{\contentsname
        \@mkboth{%
           \MakeUppercase\contentsname}{\MakeUppercase\contentsname}}%
    \@starttoc{toc}%
}
\makeatother


\makeatletter
\newenvironment{tablehere}
{\def\@captype{table}}{}
\newenvironment{figurehere}
{\def\@captype{figure}}{}
\makeatother


 \definecolor{shadecolor}{named}{AliceBlue}

\newtcolorbox{mybox}{colback=gray!25,
boxrule=0pt,arc=0pt,boxsep=2pt,left=2pt,right=2pt,leftrule=2pt, rightrule=2pt}

\newcommand{\Think}[1]{
\begin{mybox}
\includegraphics[width=0.8cm]{Images/Think.png} \enskip{}
 \textbf{Think about it.}\\#1
\end{mybox}
}


% \newcounter{appliedmechanics}
%\def\theappliedmechanics{\arabic{appliedmechanics}}
%\newenvironment{appliedmechanics}[2][]{\begin{small}\begin{shaded}\refstepcounter%{appliedmechanics} \par\medskip\noindent%
%   \textbf{Applied Mechanics~\theappliedmechanics #1: #2
%   \vspace{0.1cm} \hrule \vspace{0.1cm}}
%   \rmfamily}{\medskip \end{shaded}\end{small}}
   
 
\usepackage{fancyhdr}
\pagestyle{fancy}
\fancyhead{} % clear all header fields
\fancyhead[RO,LE]{\thepage}
\fancyhead[CE,CO]{\small\textsf{Core Physics II: Oscillations, Waves \& Fields}}
\fancyfoot{} % clear all footer fields
\fancyfoot[L]{}
\fancyfoot[CO,CE]{}
\fancyfoot[R]{\hyperlink{page.1}{\small{\color{SteelBlue}Table of Contents}}}
\renewcommand{\headrulewidth}{0pt}
\renewcommand{\footrulewidth}{0pt}


\setlist[description]{%
  font={\itshape}, % set the label font
%  font={\bfseries\sffamily\color{red}}, % if colour is needed
}

\newcommand{\highlight}[1]{\textsf{\textbf{\small #1}}}

%\newcommand{\mechanics}[1]{{\color{SteelBlue}#1}}

\newcommand{\exref}[2][Exercise~]{#1\ref{#2}}
\newcommand{\secref}[2][Section~]{#1\ref{#2}}
\newcommand{\figref}[2][\figurename~]{#1\ref{#2}}

\newlength\mytemplen
\newsavebox\mytempbox

\makeatletter
\newcommand\mybluebox{%
    \@ifnextchar[%]
       {\@mybluebox}%
       {\@mybluebox[0pt]}}

\def\@mybluebox[#1]{%
    \@ifnextchar[%]
       {\@@mybluebox[#1]}%
       {\@@mybluebox[#1][0pt]}}

\def\@@mybluebox[#1][#2]#3{
    \sbox\mytempbox{#3}%
    \mytemplen\ht\mytempbox
    \advance\mytemplen #1\relax
    \ht\mytempbox\mytemplen
    \mytemplen\dp\mytempbox
    \advance\mytemplen #2\relax
    \dp\mytempbox\mytemplen
    \colorbox{Linen}{\hspace{1em}\usebox{\mytempbox}\hspace{1em}}}

\makeatother

\newcommand{\poor}{\color{Red}poor}
\newcommand{\ok}{\color{Orange}ok}
\newcommand{\good}{\color{OliveGreen}good}

% For vectors
\renewcommand{\a}{\mathbf{a}}
\renewcommand{\b}{\mathbf{b}}
\renewcommand{\c}{\mathbf{c}}
\newcommand{\x}{\mathbf{x}}
\newcommand{\y}{\mathbf{y}}
\newcommand{\z}{\mathbf{z}}
\renewcommand{\r}{\mathbf{r}}
\renewcommand{\u}{\mathbf{u}}
\renewcommand{\v}{\mathbf{v}}
\newcommand{\kk}{\mathbf{k}}
\newcommand{\p}{\mathbf{p}}
\newcommand{\m}{\mathbf{m}}
\newcommand{\s}{\mathbf{s}}

\newcommand{\n}{\mathbf{n}}

% For differentials
\renewcommand{\d}{\mathrm{d}}
% For exponentials? 
\newcommand{\e}{\mathrm{e}}

% For vectors
\newcommand{\A}{\mathbf{A}}
\newcommand{\B}{\mathbf{B}}
\newcommand{\D}{\mathbf{D}}
\newcommand{\E}{\mathbf{E}}
\newcommand{\J}{\mathbf{J}}
\newcommand{\F}{\mathbf{F}}
\newcommand{\M}{\mathbf{M}}
\renewcommand{\H}{\mathbf{H}}
\renewcommand{\P}{\mathbf{P}}
\newcommand{\R}{\mathbf{R}}
\renewcommand{\l}{\mathbf{l}}
\renewcommand{\S}{\mathbf{S}}
\renewcommand{\L}{\mathbf{L}}
\newcommand{\Q}{\mathbf{Q}}


% For unit vectors
\renewcommand{\i}{\mathbf{i}}
\renewcommand{\j}{\mathbf{j}}
\renewcommand{\k}{\mathbf{k}}

%\DeclareMathOperator{\sinc}{sinc}

%\numberwithin{equation}{section}

%\setcounter{tocdepth}{2}

%\DeclareUnicodeCharacter{2061}{}
%\DeclareUnicodeCharacter{2003}{}